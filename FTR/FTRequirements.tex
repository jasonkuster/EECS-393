\documentclass[pdftex,12pt,letter]{article}
\usepackage{fancyhdr}
\usepackage{enumerate}
\usepackage{tabularx}
\usepackage{graphicx}
\usepackage{array}
\usepackage[justification=justified,singlelinecheck=false]{caption}
\usepackage{placeins}
\pagestyle{fancy}
\makeatletter
  \renewcommand\@seccntformat[1]{\csname the#1\endcsname.\quad}
\makeatother

\newcolumntype {Y}{ >{\raggedright \arraybackslash }X}
\newcommand{\HRule}{\rule{\linewidth}{0.5mm}}
\captionsetup{labelformat=empty}

\begin{document}

\begin{titlepage}
\begin{flushright}
\HRule \\[0.4cm]
{ \bfseries
{\huge Functional Test Requirements\\[1cm]}
{\Large for\\[1cm]}
{\huge CWRUtility\large\\[4cm]}
{\large Prepared by\\Jason Kuster, Stuart Long, and William Ordiway\\[1cm]
Version 1.0\\[1cm]
KOALAA Development\\[1cm]
November 12, 2012}}
\end{flushright}
\end{titlepage}
\tableofcontents{}
\begin{table}[!t]
\caption*{\bfseries Revision History}
\begin{tabularx}{\textwidth }[t]{|l|Y|Y|l|}
\hline
\bfseries Name & \bfseries Date & \bfseries Reasons for Change & \bfseries Version \\ \hline
Long & 9/22/2012 & Initial Draft & 1.0a\\
\hline
\end{tabularx}
\end{table}
\FloatBarrier
\newpage
\clearpage
\section{Main Page}
\begin{enumerate}[1.]
\item When opened, the application will display the main page.
\item From the main page, the user will be able to select any of the rest of the features of the application.
\item When the user selects a feature, the application will switch from the main page to that feature.
\item If a user "sets defaults" in either the \emph{eSuds} or \emph{NextBus} feature, the main page will accurately display the appropriate information next to the respective feature. (See the \emph{eSuds} and \emph{NextBus} for more information)
\item From any of the other features, if the user hits the "back" button, the user will be returned to the main page.
\end{enumerate}
\section{NextBus}
\section{Campus Map}
\begin{enumerate}[1.]
\item The user will be able to either view a "bing" map of the CWRU campus or an offical CWRU map of the CWRU campus.
\item The user will be able to swipe laterally to switch between the two maps.
\item On the "bing" map, the user will be able to zoom-in by putting two fingers on the phone and moving them away from each other.
\item On the "bing" map, the user will be able to zoom-out by putting two fingers on the phone and moving them towards each other.
\item On the "bing" map, the user will be able to pan the map by placing a single finger on the phone and moving it parallel to the direction the user wishes to pan.
\item The "bing" map will display a campus outline and several campus landmarks using pushpins.
\item The user will be able to hide this outline and these pushpins by tapping the "toggle info" button.
\item Once hidden the user will be able to display the outline and pushpins by tapping the "toggle info" button again.
\item The user will be able to tap the "my location" button, and the application will display the user's current location on the "bing" map.
\item The "toggle info" button will also affect the visibility of the pushpin representing the user's current location.
\item On the official CWRU map, the user will be able to pan around the map at will in the same manner as on the "bing" map.
\item The user will be able to return to the main page by hitting the phone's "back" button.
\end{enumerate}
\section{eSuds}
\section{Directory}
\section{Menus}
\section{Case News}
\begin{enumerate}[1.]
\item The user will be able to view an RSS feed from both \emph{The Observer} and \emph{The Case Daily}. 
\item The user will be able to switch between view stories from either news source by swiping laterally on the phone.
\item The user will be able to tap on a news story and the application will switch to a new page that displays the story in its entirety.
\item On being opened, this feature will automatically refresh the feed of the current RSS feed.
\item The user will be able to tap the "refresh feed" button to manually refresh the RSS feed of the current feed.
\item While feeds are being refreshed, the user will see a progress bar at the top of the application.
\item The user will be able to return to the main page by hitting the phone's "back" button.
\end{enumerate}
\lfoot{\includegraphics[height=1cm]{DarkKoala.png}}
\end{document}