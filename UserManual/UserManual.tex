\documentclass[pdftex,12pt,letter]{article}
\usepackage{fancyhdr}
\usepackage{enumerate}
\usepackage{tabularx}
\usepackage{graphicx}
\usepackage{array}
\usepackage{placeins}
\usepackage[justification=justified,singlelinecheck=false]{caption}
\pagestyle{fancy}
\makeatletter
  \renewcommand\@seccntformat[1]{\csname the#1\endcsname.\quad}
\makeatother

\newcolumntype {Y}{ >{\raggedright \arraybackslash }X}
\newcommand{\HRule}{\rule{\linewidth}{0.5mm}}
\captionsetup{labelformat=empty}

\begin{document}

\begin{titlepage}
\begin{flushright}
\HRule \\[0.4cm]
{ \bfseries
{\huge  User Manual\\[1cm]}
{\Large for\\[1cm]}
{\huge CWRUtility\large}\\[4cm]}
{\large KOALAA Development\\[1cm]December 7, 2012}
\end{flushright}
\end{titlepage}
\tableofcontents{}

\newpage
\section{Overview}
\subsection{Summary}
For students, faculty, or visitors to Case Western Reserve University campus who struggle with using the myriad of resources CWRU provides, CWRUtility is an app which will solve these problems. It is an elegantly designed mobile application which will provide useful information and centralize the services students use most.
\begin{enumerate}[FE-1:]
\item Main Page: Displays a start page with information on and links to the other features.
\item NextBus: Integrates with NextBus, Inc. to provide a schedule for the "greenie" system.
\item Campus Map: Provides map of CWRU campus.
\item eSuds: Integrates with the e-Suds online system which provides laundry information.
\item Directory: Displays locations, phone numbers and descriptions for Case's on campus resources.
\item Menus: Displays the dinning selections for both major dining halls.
\item Case News: Integrates with \emph{The Case Daily} and \emph{The Observer} to display daily news.
\end{enumerate} 
\subsection{Using Internet}
Window's Mobile devices are equipped with  internet connectivity that will be implemented in the NextBus, Campus Map, eSuds, Menus, and Case News features. If the user can not connect to the internet or one of the online resources responsible for providing data to one of these features goes down the feature and/or application may not function correctly or completely. 

\section{Main Page}
The Main Page is the first page you will view when you open the CWRUtility application. On it is a list of 6 resources: Next Bus, Campus Map, eSuds, Directory, Menus, and Case News. Tapping any feature name on the title page will take the user to that feature. Next Bus and eSuds may display information next to their name based on the settings the user can establish when using those features as described below. 

\section{Next Bus}
\subsection{Overview}
This feature displays the 3 soonest arrival times for a bus route, in a direction, at a specified stop. The user can return to the Main Page by hitting the back arrow at the bottom of the phone.
\subsection{Selectioning Stops}
To get the schedule of a "greenie" bus relevant to your interest customize which route, direction, and stop you want to view information for. To do this tap, using your finger, the grey boxes which are set to "Circle Link", "To Circle Station", and "Cleveland Museum of Art" by default. This will change your phone's screen to a list of the available options for that selection. After making selections for route, direction, and stop get estimates for when the bus will arrive by tapping the "Go" button. This will update the 3 numbers near the top of the screen: these numbers describe the how soon, measured in minutes, until that the bus will visit that stop in the selected direction. These numbers will update as time passes. However if you revise what route, direction, or stop you would like to view information for you will need to hit the go button again. Please note that not all bus routes run all day, for example the North Loop only runs after 5:30 PM, and so if the route you wish to view is not currently running instead of loading a countdown for bus arrivals "No Prediction Available" will be loaded at the top of the screen. The user can return to the Main Page by hitting the back arrow at the bottom of the phone.
\subsection{Set Default}
Below the "Go" button is a Star with a '+' sign in the lower corner. This button may be used to set a default route, direction, and stop. By choosing a default bus route, direction, and stop these will be what route, direction, and stop are automatically set to instead of  "Circle Link", "To Circle Station", and "Cleveland Museum of Art" when you reopen the app. Additionally the arrival times for these busses will be displayed on the Main Page as well right next to the Next Bus feature name.

\section{Campus Map}
\subsection{Overview}
This feature displays a Bing Map of campus and the area around campus and a campus provided map of campus. The map can be zoomed in on by pinching two fingers together on the screen. Zooming out is down by doing this motion in reverse. The user may also use fingers in a swipping motion to switch back and forth between the Road Map and the CWRU Map, or may tap the names of each map at the top of the map to alternate. The user can return to the Main Page by hitting the back arrow at the bottom of the phone.
\subsection{Road Map}
Upon loading the Campus Map feature, the user will be presented with a slice from Bing Maps. Outlined in black is the edges of the Case Western Reserve University campus. The on campus residential areas ("North Side" and "South Side") are labeled with a flag, as are the two academic building clusters ("The Quad" and "Mather Quad"). At the bottom of the map are two buttons that the user can engage by tapping with a finger. 
\subsection{Eyeball Icon}
Tapping the eyeball icon will toggle the outline of campus and the flags designating the notable areas around campus. 
\subsection{Co-Ordinates Icon}
Tapping the icon of an arrow pointing down will obtain the users coordinates, zoom the map to the user's location, and mark the user's location with a red marker. Using the eyeball icon to toggle the visibility of the campus outline and markers will also toggle the users position marker.
\subsection{CWRU Map}
This map is a resource provided by the campus which labels every building on campus and has a more accurate definitions for where buildings are exactly in relation to eachother. The flags and location marker features are not available on this map.

\section{eSuds}
\subsection{Overview}
eSuds will present the user with a list of washers and dryers in a building and whether or not they are currently contain somebody elses laundry or if they are available for the user to use. This is done by listing whether a machine is a washer or dryer in the "Type" column, listing whether the machine is available by stating "available" in green text under status or stating "In Use" in red. If a machine is in use a timer for when the machine will be free for use is displayed in the Time column. The number of the machine is listed in the left most column titled "M\#". The user can scroll through the machines by swipping up and down, in the case that there are more machines than the user can view on the screen at one time. The user can return to the Main Page by hitting the back arrow at the bottom of the phone.
\subsection{Building}
By default this feature gives information for the Alumni building. By tapping the grey box though the user is brought to a list of buildings with washers and dryers which the user can then select from to get information about the availability of that buildings laundry machines. 
\subsection{Set Default}
Below the list of machines is a button with Star and '+' sign in the lower corner. This button may be used to set the current building as the default to get laundry information about. Selecting a default will make it so on the Main Page the number of available washers and dryers will be displayed next to the eSuds name.

\section{Directory}
\subsection{Overview}
The Directory feature displays a list of resources available on campus in (mostly) alphabetical order. By tapping the name of a resource it will reveal a phone number, address, and a brief description of the objective of that resource. Tapping the name of an already opened resource will close that resource collapsing the phone number, address, and description. Additionally, by tapping a resource while another resource is open will collapse the previously opened resource. Tapping the phone number and address of a resource have additional functions. The user can return to the Main Page by hitting the back arrow at the bottom of the phone.
\subsection{Phone}
By tapping the phone number of a resource, a windows phone will prompt the user with a dialog box that double checks that the user wants to connect the call. By selecting "call" the phone will dial the resource and you will be connected in a phone call. By selecting "don't call" the call will not be made and the user will return to the Directory.
\subsection{Address}
By tapping an address the user will be taken to the Road Map, with a flag designating the location of the resource. The user can return to the Directory by hitting the back arrow at the bottom of the phone.

\section{Menus}
\subsection{Overview}
This feature will load the dinning selections available at Fribley and Leutner for the current day by default, but the dinning selections for any other day in the week can be obtained. "Main Courses" are in whatever the user's phone Accent Color is, while side dishes are in light grey and the meal (Breakfast, Lunch, Dinner) is in white. The user may also use fingers in a swipping motion to switch back and forth between Fribley and the Leutner, or may tap the names of each dinning hall at the screen to alternate. The user can scroll through menus by swipping up and down with their fingers. The user can return to the Main Page by hitting the back arrow at the bottom of the phone.
\subsection{Date Selection}
By tapping the grey bar at the bottom of the phone, the user may select what day of the week they want to view the menus for.

\section{Case News}
\subsection{Overview}
This feature will load  \emph{The Case Daily} and \emph{The Observer} to display daily news. The user may also use fingers in a swipping motion to switch back and forth between  \emph{The Case Daily} and \emph{The Observer}, or may tap the names of the publications at the top of the screen to alternate. The user can scroll through articles by swipping up and down with their fingers. The user can return to the Main Page by hitting the back arrow at the bottom of the phone. 
\subsection{Articles}
Article titles are displayed in the user's phone Accent Color, while the beginning of the article is below in white and in grey below that is the publication date. Clicking any of these will take the user to the full arcticle on the publications respective website.
\subsection{Refresh Feed}
The button at the bottom of the screen can be used to check if any new article have been published since the user has opened the application. Articles are listed chronologically, with the most recent articles listed at the top of the screen.

\end{document}