\documentclass[pdftex,12pt,letter]{article}
\usepackage{fancyhdr}
\usepackage{enumerate}
\usepackage{tabularx}
\usepackage{graphicx}
\usepackage{array}
\usepackage[justification=justified,singlelinecheck=false]{caption}
\usepackage{placeins}
\pagestyle{fancy}
\makeatletter
  \renewcommand\@seccntformat[1]{\csname the#1\endcsname.\quad}
\makeatother

\newcolumntype {Y}{ >{\raggedright \arraybackslash }X}
\newcommand{\HRule}{\rule{\linewidth}{0.5mm}}
\captionsetup{labelformat=empty}

\begin{document}

\begin{titlepage}
\begin{flushright}
\HRule \\[0.4cm]
{ \bfseries
{\huge Design Document\\[1cm]}
{\Large for\\[1cm]}
{\huge CWRUtility\large\\[4cm]}
{\large Prepared by\\Jason Kuster, Stuart Long, and William Ordiway\\[1cm]
Version 1.0 initial\\[1cm]
KOALAA Development\\[1cm]
October 11, 2012}}
\end{flushright}
\end{titlepage}
\tableofcontents{}
\begin{table}[!t]
\caption*{\bfseries Revision History}
\begin{tabularx}{\textwidth }[t]{|l|Y|Y|l|}
\hline
\bfseries Name & \bfseries Date & \bfseries Reasons for Change & \bfseries Version \\ \hline
Long & 10/6/2012 & Initial Outline & 1.0 initial\\
Kust, Long, Ordiway & 10/10/12 & Initial Draft & 1.0\\
\hline
\end{tabularx}
\end{table}
\FloatBarrier
\newpage
\clearpage
\section{Overview}
\subsection{Overall Design}
The \emph{CWRUtility} application is essentially a collection of CWRU-related features that allows for easy, centralized access to each of the features. The complete list of features and their description can be found in the \emph{CWRUtility} SRS. Since this software system can be easily broken down into separate, uncoupled features, the design of the overall system is also broken down. Each feature will have it's own design, documented below in the "Features" section, that will operate independently of the other features. The only exception to this rule is the "StartPage" feature, which can also be seen as the overall application manager. Explained in detail below, the "StartPage" feature will be responsible for navigating the system to the different features and controlling any communication between the features. The figure below shows the various components of the system and how they are linked. This diagram could be seen as the highest level of design for the system.
\includegraphics[width=120mm]{OverallCD.png}
\subsection{User Interface}
The implementation of any user interfaces on the Windows Phone 7/8 platform is done using the eXtensible Application Markup Language (XAML). Essentially, XAML can be used as a layout language similar to HTML. XAML provides a straightforward way to generate, lay out, and populate the on-screen elements. Data population will be done by using Model View View-Model (MVVM), a data-binding interface built into the Windows Phone SDK. The data fields will be laid out in XAML, which will comprise the view; the code to map the data to the view is contained in the view-model, and the raw data will be stored in application storage.It is important to note that a markup language is very dissimilar to a programming language as it works through simply creating a series of attributes and assigning values to them. The Windows Phone OS actually handles creating the UI based on these attributes, therefore this software system does not have to.
\\
\section{Features}
\subsection{StartPage}
The "StartPage" feature represents both the manager for the rest of the application and the feature the user is taken to on application start-up. This feature actually comprises two different application pages for the "At-a-glance" page and one for the page listing the full list of features.
\subsubsection{Class Diagram}
\includegraphics[width=120mm]{StartPageCD.png}
\subsubsection{StartPage}
The StartPage class will be the main class in this feature. It will be the feature that opens when the user starts the application, an operation that is handled by the XAML (UI) file. The page will display the application title at the top, then the feature title below that, and finally a scrollable list of featurePanels. Each featurePanel will represent one of the other features within the application and will display the feature title as well as a potentially "live" description for that feature. For example, one of the features in this feature panel will be the "nextBus" feature, and its feature panel will display the next shuttle time for the user's last selected stop. This page will display all of the applications features in a scrollable list. If the user clicks on a feature panel, the application will switch the user to that feature.
\subsubsection{FeaturePanel}
There will be an instance of feature panel for every feature in the application. A featurePanel will store the title of the feature and a description for that feature. Since descriptions can be "live", or constantly updating its information, some of the instances of featurePanel will have to override the getDescription() method in order to properly display the correct up-to-date information. The "StartPage" class will send a list of these FeaturePanels to the UI manager so that they can be displayed to the user. The switchToFeature() will be called when the user selects a featurePanel from the displayed list. The application then creates a new instance of the selected feature and switches to that feature.
\subsubsection{UIManager}
The XAML file responsible for laying out the components of this feature for display to the user. This feature will have the title of the application and then the title of this feature. Under the titles, the majority of the feature will be taken up by a scrollable list of featurePanels. The XAML file will get the list of featurePanels by calling the getFeatures() method in the StartPage class. 
\subsection{Map}
The "Map" feature will allow the users to view a map of the Case Western Reserve University Campus. The feature will use the interactive map controller provided by the Windows Phone SDK. This map controller is the same map controller used for the default Windows Phone 7 map application. Since the map controller is already provided by the Windows Phone SDK, this feature will have a simple design.
\subsubsection{Class Diagram}
\includegraphics[width=120mm]{MapCD.png}
\subsubsection{SilverlightMapControl}
The SilverlightMapControl class is part of a third party library provided by Microsoft Corporation. It provides basic map controls such as zooming in, zooming out, and panning for any provided map. The MapManager will simply send a saved map of the CWRU campus to the map controller.
\subsubsection{MapManager}
The MapManager class is responsible for handling the other two classes: the SilverlightMapControl and the UI manager. It will have a saved copy of the CWRU map that it will send to the SilverlightMapControl. It will also be responsible for sending the SilverlightMapControl to the UI manager so that the UI manager can display the map.
\subsubsection{UIManager}
The UIManager for this feature will not be a true class, since all UI for Windows Phone is done using XAML files. Still, this manager will be responsible for laying out the components of the feature. Namely, it will have to layout the title of the application at the top, the name of this feature below that, and the actual map will take up most of the application. The XAML file will be sent the map controller (along with that map controller's view, i.e. the map)  by the MapManager.
%william's section
\subsection{Directory}
The "Directory" feature will allow users to view the location, hours, phone number, and a description of the wide assortment of Case Western Reserve University campus resources. This feature will be simple in design because besides presenting preloaded information its only function is to facilitate calling campus resources 
\subsubsection{Class Diagram}
\begin{flushleft}
\includegraphics[width=120mm]{DirectoryCD.png}
\end{flushleft}
\subsubsection{Resource}
Each campus resource that will be represented in our application will be represented as an object in the resource class possessing a Phone Number, an Address, open hours, and a brief description about what the resource can be used for. The address, open hours, and description will all be simple text variables which are displayed on screen. However the phone number variable is tied to the PlaceCall() operation. By tapping the phone number of a resource the user’s phone will place a call to that number. This information about every campus resource will be manually loaded into the app by the developers and in the event that a resource was omitted or a new resource is offered by the campus this feature will have to be updated to include that resource. 
\subsubsection{Directory}
The directory class is used to maintain a variable List which contains a number of resource objects, each of which represent a different campus resource. That list will be displayed on screen with the GetList() operation via the UI manager once the Directory feature is selected from the start page. Selecting a campus resource by touching it on the screen with your finger will expand information about that resource with the ExpandResource() operation and shut the previous, if any, selected resource with a MinimizeResource() operation. Selecting an already open resource will also minimize it.
\subsubsection{UI Manager}
The UIManager for this feature will not be a true class, since all UI for Windows Phone is done using XAML files. Still, this manager will be responsible for laying out the components of the feature. Namely, it will have to lay out the title of the application at the top, and the name of this feature below that. Taking up the majority of the screen though will be a scrollable alphabetized list of campus resources. The XAML File will receive the GetList() operation which dictates the layout of the list of resources.
\subsection{NextBus}
The "NextBus" feature will allow the users to interface with the NextBus, Inc. bus times prediction website. This feature will offer users the ability to select bus routes, bus direction, and stops in order to facilitate the use of the "Greenie" bus system on campus. The feature's main page will be a single Windows Phone-style page with three drop-down selection boxes and a "Go" button. There will be a field, initially blank, which will populate with the next prediction times (max 3). Additionally, the NextBus section will feature on the main page, where it will provide a small icon which contains the stop which the user has marked as a "favorite".
\subsubsection{Class Diagram}
\begin{flushleft}
\includegraphics[width=120mm]{nextbusCD.png}
\end{flushleft}
\subsubsection{BusScheduleManager}
The BusScheduleManager class handles retrieval and formatting of the NextBus data. Since Case Western has not purchased the feed option from NextBus, aggregating data will require scraping the HTML that NextBus generates for each bus stop once it is requested. This class will also handle the drop-down menus, populating them with the requisite items once route and direction have been selected. There are two listener methods which will populate the next dropdown in the list when the previous item is successfully selected. Additionally, there is a method which runs when the "Go" button is pressed, which will pull the requisite info from the NextBus website and display it on the screen.
\subsubsection{UIManager}
The UIManager for this feature will not be a true class, since all UI for Windows Phone is done using XAML files. Still, this manager will be responsible for laying out the components of the feature. Namely, it will have to lay out the title of the application at the top, and the name of this feature below that. There will be three dropdown menus on the top of the screen - one for Route, Direction, and Stop. There will be a "Go" button underneath which will initiate the prediction request. Additionally, there will be three text boxes, initially empty, which will populate once the predictions are acquired.
%\subsection{Schedule}
%The "Schedule" feature will allow users to add classes or custom events to a saved schedule on the system. This feature will require three classes: a schedule event, a schedule event manager to handle schedule events, and a UI manager for the schedule.
\lfoot{\includegraphics[height=1cm]{DarkKoala.png}}

%\subsection{Class Diagram}
%\includegraphics[width=80mm]{ScheduleEventsCD.png}
%\figurename{Schedule Class Diagram}
%\subsection{Class Descriptions}
%The following class descriptions describe the above class diagram.
%\subsubsection{ScheduleEvent}
\end{document}
