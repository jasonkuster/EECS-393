\documentclass[pdftex,12pt,letter]{article}
\usepackage{fancyhdr}
\usepackage{enumerate}
\usepackage{tabularx}
\usepackage{graphicx}
\usepackage{array}
\usepackage[justification=justified,singlelinecheck=false]{caption}
\usepackage{placeins}
\pagestyle{fancy}
\makeatletter
  \renewcommand\@seccntformat[1]{\csname the#1\endcsname.\quad}
\makeatother

\newcolumntype {Y}{ >{\raggedright \arraybackslash }X}
\newcommand{\HRule}{\rule{\linewidth}{0.5mm}}
\captionsetup{labelformat=empty}

\begin{document}

\begin{titlepage}
\begin{flushright}
\HRule \\[0.4cm]
{ \bfseries
{\huge Software Requirements Specification\\[1cm]}
{\Large for\\[1cm]}
{\huge CWRUtility\large{\footnote[1]{Working title}}\\[4cm]}
{\large Prepared by\\Jason Kuster, Stuart Long, and William Ordivay\\[1cm]
Version 1.0\\[1cm]
KOALAA Development\\[1cm]
October 1, 2012}}
\end{flushright}
\end{titlepage}
\tableofcontents{}
\vspace{5cm}
\begin{table}[h]
\caption*{\bfseries Revision History}
\begin{tabularx}{\textwidth }[t]{|l|Y|Y|Y|}
\hline
\bfseries Name & \bfseries Date & \bfseries Reasons for Change & \bfseries Version \\ \hline
Kuster, Long, Ordiway & 9/22/2012 & Initial Draft & 1.0 draft 1\\ \hline
\end{tabularx}
\end{table}
\newpage
\section{Introduction}
\subsection{Purpose}
This Software Requirements Specification describes the software functional and nonfunctional requirements for release 1.0 of CWRUtility. This specification document is intended for the solu use of the members of the project team that will implement and verify the correction functioning of the system. Unless otherwise specified, all requirements documented here are high priority and committed for release 1.0. 

\subsection{Project Scope and Product Features}
CWRUtility will allow CWRU students, faculty, and staff to easily access CWRU services and information in an easy-to-use way. A detailed project description is available in the \emph{CWRUTility Vision and Scope Document}[1]. The section in that document titled "Scope of initial and subsequent releases" lists the features that are scheduled for full or partial implementation in this release.

\subsection{References}
\begin{enumerate}[1.]
\item Kuster et al. \emph{Vision and Scope Document for CWRUtility}, https://github.com/jasonkuster/EECS-393/blob/master/V%26S/VisionAndScope.pdf
\end{enumerate}

\section{Overall Description}
\subsection{Product Perspective}
CWRUtility is a new system that replaces the need for cumbersome navigation of all of the CWRU services in favor of a streamlined and custom-built mobile experience. The context diagram in Figure 1 illustrates the external entities and system interfaces for release 1.0. The system is expected to evolve over two releases, ultimately encompassing all of the services CWRU students, faculty, and staff use every day.
\subsection{User Classes and Characteristics}
\noindent Users: CWRUtility users are CWRU students, faculty, or staff members who wish to use our app in order to access CWRU services. The number of people at CWRU who possess Windows Phone 7/8 devices is is unknown, but can be estimated in the area of 200 to 400 assuming that about half to 3/4 of the community possesses smartphones and the current figures put Windows Phone's marketshare at 3.5\%. Our app will probably be used daily by 75\% of installed users.\\\\
CWRU Services:  The CWRU services are the services enumerated in Diagram 1 with which CWRUtility will have active communication. These services will be accessed several times per day by each instance of the application as it pulls data from the background. Multiple users may access the same service at the same time. Downtime in a service will result in downtime for the application.
\begin{figure}[h]
\begin{center}
\includegraphics[width=110mm]{ContextDiagram.png}\\
\caption{Figure 1: \emph{Context diagram for release 1.0 of CWRUtility}}
\end{center}
\end{figure}
\subsection{Operating Environment}
\begin{enumerate}[{O}E-1:]
\item CWRUtility shall operate on the Windows Phone 7 and Windows Phone 8 operating systems.
\end{enumerate}
\subsection{Design and Implementation Constraints}
\begin{enumerate}[CO-1:]
\item Design, code, and maintenance will conform to KOALAA's coding and documentation standards.
\item All user interfaces will be laid out with XAML.
\item All code will be written in C\#.
\end{enumerate}
\subsection{User Documentation}
\begin{enumerate}[UD-1:]
\item System shall provide in-system tutorial and help documentation accessible via an easy to find button.
\end{enumerate}
\subsection{Assumptions and Dependencies}
\begin{enumerate}[{A}S-1:]
\item User's Windows Phone 7/8 device is equipped with an internet connection.
\end{enumerate}
\begin{enumerate}[DE-1:]
\item This application depends on the continuing functionality of the connected services (e.g. NextBus.com, eSuds.com, and SIS)
\end{enumerate}
\section{System Features}
\subsection{Schedule Courses}
\subsubsection{Description and Priority}
A CWRUtility User may search for CWRU courses by semester and course name. Duplicate courses in a semester are returned as a drop down list. The User may then choose to add that course to his or her current schedule. Information for CWRU courses are pulled from the Student Information System Service. The User may also delete courses from his or her current schedule. The system displays the User's schedule in a standard calendar format. Priority: High
\subsubsection{Stimulus/Response Sequences}
\begin{description}\itemsep1pt
\item[Stimulus:] User searches for a course with a specified semester and course.
\item[Response:] System queries SIS for detailed course information and relays it to user. If more than one course match the given semester and course name, results are displayed in a dropdown list.
\item[Stimulus:] User adds chosen course to schedule
\item[Response:] System plots course on user's current schedule.
\item[Stimulus:] User selects a course to remove from schedule.
\item[Response:] System removes selected course from schedule.
\end{description}
\subsubsection{Functional Requirements}
\begin{table}[!h]
\begin{tabularx}{\textwidth }[t]{|l Y|}
\hline
~&~\\
Courses.Search & The system will allow the user to search for a course based on user inputted course name and semester.\\ 
~&~\\
Courses.Search.Invalid & If the user enters an invalid course name or semester, no course will be returned by the system and it will inform the user appropriately.\\
~&~\\
Courses.Search.Return & If a course is found that matches the user criteria, the system shows the course and its information. \\
~&~\\
Courses.Search.Multiple & If more than one course matches the user criteria, the system will display the courses in a drop-down list that allows the user to select an appropriate course.\\
~&~\\
\hline
\end{tabularx}
\end{table}
\begin{table}[!h]
\begin{tabularx}{\textwidth }[t]{|l Y|}
\hline
~&~\\
Courses.Schedule & The system will display user added courses on a calendar formatted schedule.\\
~&~\\
Courses.Schedule.Add & Courses selected to be added to the schedule will be added to the schedule and placed in their appropriate time slot.\\
~&~\\
Courses.Schedule.Remove & The system will allow users to remove any and all courses from the user's schedule.\\
~&~\\
\hline
\end{tabularx}
\end{table}
\FloatBarrier
\subsection{Map}
\subsubsection{Description and Priority}
All relevant campus buildings should be accurately labelled on the map. This includes not only academic buildings but residence halls, cafeterias, and other miscellaneous CWRU-owned buildings as well.\\
Upon being opened, if the user has entered in courses, the map should predict their next destination and offer a map perspective encompassing both their current location and their projected next location.\\
The map will offer the student the ability to navigate from their current location to their next class, or to an alternate location.
\subsubsection{Stimulus/Response Sequences}
Stimulus: User opens map.\\
Response: App queries its stored information regarding whether the student has a class, then sets the starting map coordinates and zoom level accordingly.\\
Stimulus: User requests directions.\\
Response: Map provides directions from the user's current location to the specified location.\\
\subsubsection{Functional Requirements}
\begin{table}[h]
\begin{tabularx}{\textwidth }[t]{|l|Y|Y|Y|}
\hline
\bfseries Feature & \bfseries\hspace{1cm}Release 1 & \bfseries\hspace{1cm}Release 2 & \bfseries\hspace{1cm}Release 3 \\ \hline
FE-1 & Fully implemented & ~ & ~ \\ \hline
FE-2 & Basic map & Integrated with course schedule & Provides directions \\ \hline
FE-3 & Basic functionality with NextBus & Full integration with NextBus (if needed) & Integration with Map \\ \hline
FE-4 & Fully implemented & ~ & ~ \\ \hline
FE-5 & Not implemented & Fully implemented & ~ \\ \hline
FE-6 & Fully implemented & ~ & ~ \\ \hline
FE-7 & Not implemented & Fully implemented & ~ \\ \hline
FE-8 & Not implemented & Basic information displays & Integrates with phone's notification system \\ \hline
FE-9 & Table of events & ~ & Integrated with course schedule \\ \hline
FE-10 & Not implemented & Fully implemented & ~ \\
\hline
\end{tabularx}
\end{table}
\subsection{Nextbus}
\subsubsection{Description and Priority}
A user will be presented with the Case Western University nextbus webpage which presents an active timer until busses arrive. Users can specify which bus route, which stop, and which direction 
\subsection{Directory}
\subsubsection{Description and Priority}
The system will include a directory that lists hours, location, and phone numbers of campus resources. The user will be able to browse the directory and select a resource to see the above information about it. The directory will be designed such that more resources can be added easily. 
\subsubsection{Stimulus/Response Sequences}
\begin{description}\itemsep1pt
\item[Stimulus:] Users selects a service from the directory.
\item[Response:] System queries its own database/backend and displays the stored information about that service to the user.
\item[Stimulus:] Users clicks on a the phone number of some selected service
\item[Response:] System attempts to place a mobile phone call using selected phone number
\end{description}
\subsubsection{Functional Requirements}
\begin{table}[!h]
\begin{tabularx}{\textwidth }[t]{|l Y|}
\hline
~&~\\
Directory.Browse & The system will allow the user to browse through CWRU services\\ 
~&~\\
Directory.Browse.Select & If the user selects a service from the directory, system displays descriptive information about the service, e.g. hours. location, and phone number.\\
~&~\\
\hline
~&~\\
Directory.Call & After selecting the service, if the user clicks that service's phone number, the cell phone the system is implemented dials that phone number\\
~&~\\
\hline
\end{tabularx}
\end{table}
\FloatBarrier
\subsection{Menu}
\subsubsection{Description and Priority}
The system will display the semester menu's for both major campus dining halls, Fribley Dining Hall and Leutner Dining Hall. The system will layout of the menus in a tabular format, so at anytime the menu portion of the application will either display the Leutner menu or the Fribley menu, depending on which dining is selected.
\subsubsection{Stimulus/Response Sequences}
\begin{description}\itemsep1pt
\item[Stimulus:] User selects a dining hall whose menu the user wants to view.
\item[Response:] The system displays the menu for the specified dining hall.
\end{description}
\subsubsection{Functional Requirements}
\begin{table}[!h]
\begin{tabularx}{\textwidth }[t]{|l Y|}
\hline
~&~\\
Menu.Display & System will display the menu for a selected campus dining hall.\\ 
~&~\\
Menu.Display.Select & System will switch displayed menu based on which tab is selcted\\
~&~\\
\hline
\end{tabularx}
\end{table}
\FloatBarrier
\lfoot{\includegraphics[height=1cm]{DarkKoala.png}}
\end{document}